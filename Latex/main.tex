\documentclass[a4paper]{article}
\usepackage[utf8]{inputenc}
\usepackage[slovene]{babel}
\usepackage{graphicx}
\usepackage{hyperref}
\usepackage[nottoc]{tocbibind}
\usepackage{caption}
\usepackage{subcaption}
\usepackage{amsmath}
\usepackage{ dsfont }
\usepackage{siunitx}
\usepackage{multimedia}
\usepackage[table,xcdraw]{xcolor}
\setlength\parindent{0pt}

\newcommand{\ddd}{\mathrm{d}}
\newcommand\myworries[1]{\textcolor{red}{#1}}
\newcommand{\Dd}[3][{}]{\frac{\ddd^{#1} #2}{\ddd #3^{#1}}}

\begin{document}
\begin{titlepage}
    \begin{center}
        \includegraphics[]{logo.png}
        \vspace*{3cm}
        
        \Huge
        \textbf{Naključna števila in integracije z metodo Monte Carlo}
        
        \vspace{0.5cm}
        \large
        7. naloga pri Modelski Analizi 1

        \vspace{4.5cm}
        
        \textbf{Avtor:} Marko Urbanč (28232019)\ \\
        \textbf{Predavatelj:} prof. dr. Simon Širca\ \\
        \textbf{Asistent:} doc. dr. Miha Mihovilovič\ \\
        
        \vspace{1.8cm}
        
        \large
        23.11.2023
    \end{center}
\end{titlepage}
\tableofcontents
\newpage
\section{Uvod}
Danes bojo pomembna naključna števila, saj bomo z metodo Monte Carlo računali integrale. 
Dobro je, da se zavedamo dejstva, da naključna števila, ki nam jih da recimo \texttt{np.random}
niso dejansko naključna, ampak so generirana po nekem algoritmu, ki je determinističen. Težko (beri: nemogoče) je
računalniku dati natančna navodila o tem kako naj nekaj nenatančno oz. poljubno naredi. Zato se poslužujemo
nekaterih algoritmov, ki nam pomagajo pri generiranju naključnih števil. V tem primeru bomo uporabili
\texttt{numpy.random.uniform}, ki nam vrne naključno število iz enakomerne porazdelitve na intervalu $[0,1)$.
To my surprise je, da \texttt{numpy} uporablja Mersenne Twister algoritem. In hindsight bi lahko vzel tudi kakšen 
preprostejši algoritem, ki bi bil hitrejši, čeprav \texttt{numpy} uporablja nekaj trikov, da je hitrejši.

Pri Monte Carlo integraciji naključno izbiramo točke v prostoru, kjer imamo neko omejeno območje, ki ga integriramo.
Recimo, da je to območje $[a,b] \times [c,d]$, kjer smo od tega še nekaj odrezali, da dobimo trikotnik v profilu. V takem primeru 
bi izbirali naključne točke $(x,y)$, kjer je $x \in [a,b]$ in $y \in [c,d]$ in vsakič, ko izberemo točko, preverimo, če je
znotraj območja, ki ga integriramo (v temu preprostemu primeru, ko nekaj odrežemo, preverimo ali je točka znotraj trikotne prizme).



\section{Naloga}

\section{Opis reševanja}

\section{Rezultati}


\section{Komentarji in izboljšave}

\newpage
\bibliographystyle{unsrt}
\bibliography{sources}
\end{document}
